\documentclass[12pt]{article}
\usepackage{amsmath}


\begin{document}


\title{Hello world}
\author{Tom Chothia}

\maketitle



\section{Introduction}

Latex is a type setting system, you compile tex files into pdfs. This file has a few examples, it's easy to find anything else you need on google.

\section{ Background}
References are done using a bib file. e.g. this \cite{bhargavan2016practical}, this \cite{bantin2010} or this \cite{brualdi2010introductory}.


\section{Figures and Tables}

Tables can be very useful, see Figure \ref{fig:travel}.

\begin{figure}[t]
\begin{center}
\begin{tabular}{c | c | c |  c}
Travel Method & Time & Cost & Ease \\
\hline
Bus & 40mins & \pounds 2.40 & 3 \\
Train & 20mins & \pounds 3.40 & 2 \\
\end{tabular}
\end{center}
\caption{Methods of travel to work}
\label{fig:travel}
\end{figure}

\section{Some maths}

Math model makes type setting maths easy, e.g., $x = y^4$

  \begin{align}
    E_0 &= mc^2 \\
    E &= \frac{mc^2}{\sqrt{1-\frac{v^2}{c^2}}}
  \end{align}  
  
  \begin{align}
	E_0 &= mc^2 \\
	E &= \frac{mc^2}{\sqrt{1-\frac{v^2}{c^2}}}
  \end{align}

  
  
\bibliographystyle{plain}
\bibliography{biblio}
  
\end{document}
